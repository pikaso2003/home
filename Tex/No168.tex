\documentclass[a4j,10pt,oneside,openany]{jsbook}
%
\usepackage{amsmath,amssymb}
\usepackage{bm}
\usepackage[dvipdfmx,hiresbb]{graphicx}
\usepackage{ascmac}
\usepackage{makeidx}
%
%\makeindex
%
\newcommand{\diff}{\mathrm{d}}  %微分記号
\newcommand{\divergence}{\mathrm{div}\,}  %ダイバージェンス
\newcommand{\grad}{\mathrm{grad}\,}  %グラディエント
\newcommand{\rot}{\mathrm{rot}\,}  %ローテーション
\newcommand{\btu}{\bigtriangleup} % triangle mark
\newcommand{\mr}{\mathrm}
%
\setlength{\textwidth}{\fullwidth}
\setlength{\textheight}{44\baselineskip}
\addtolength{\textheight}{\topskip}
\setlength{\voffset}{-0.6in}
%
\title{{\Huge \textbf{Dr.Hongoの数理科学ゼミ 第168問}}\\}
\author{Yuji Hiramatsu}
\date{}
%
%
%
\begin{document}
%
%
\maketitle
%\frontmatter
%\tableofcontents
%
%
%\mainmatter

%\chapter{...}
%\begin{abstract}
%...
%...
%\end{abstract}

{\Huge 168問 解答}

\vspace{3\baselineskip}


{\Large (1)}
\\
\\
行列$A$の固有ベクトルを$X$、固有値を$\lambda$とすると、
\[ AX=\lambda X \]
であるから、$(A-\lambda E)X=0$であり、固有値$\lambda$を定める行列式を考えると、
\begin{align*}
|A-\lambda E| 	&= \begin{vmatrix} a-\lambda & b & c \\ b & c-\lambda & a \\ c & a & b-\lambda \end{vmatrix} \\[8pt]
			&= \begin{vmatrix} a+b+c-\lambda & b & c \\ a+b+c-\lambda & c-\lambda & a \\ a+b+c-\lambda & a & b-\lambda \end{vmatrix} \\[8pt]
			&= (a+b+c-\lambda) \begin{vmatrix} 1 & b & c \\ 1 & c-\lambda & a \\ 1 & a & b-\lambda \end{vmatrix} \\[8pt]
			&= (a+b+c-\lambda) \begin{vmatrix} 1 & 0 & 0 \\ 1 & c-b-\lambda & a-c \\ 1 & a-b & b-c-\lambda \end{vmatrix} \\[8pt]
			&= (a+b+c-\lambda)\bigl( \lambda^2 - (a^2+b^2+c^2-ab-bc-ca) \bigr)
\end{align*}
となる。以上から、固有値は、
\[ \lambda_{1}=\underline{a+b+c} \]
\[ \lambda_{2}=\underline{\sqrt{a^2+b^2+c^2-ab-bc-ca}} \; \; \; \lambda_{3}=\underline{-\sqrt{a^2+b^2+c^2-ab-bc-ca}} \]

\vspace{1\baselineskip}

{\Large (2)}
\\
\\
$\lambda_{1}=\lambda_{2}$となる場合を考える。\\
(1)の結果をもとに、上式の両辺を二乗して式変形を施すと、
\begin{align*}
(a+b+c)^2	&=a^2+b^2+c^2-ab-bc-ca \\[5pt]
\Leftrightarrow ab+bc+ca &= b(a+c)+ca = 0 \\[5pt]
\Leftrightarrow b^2+b(a+c)+ca &= (b+a)(b+c) =b^2 
\end{align*}
となる。
題意より、$a$、$b$、$c$は整数であるが、固有方程式は、3つの変数に対して対称性をもつので、$a>b>c$としても一般性を失わない。\\
例えば、$b=1$としてみると、$b+a=1\Leftrightarrow a=0$、$b+c=1\Leftrightarrow c=0$となり、題意に反する。\\
同様に考えて、$b=3$の時、\\
$a+b=9$、$b+c=1$となり、
\[ \underline{(a,b,c)=(6,3,-2)} \]
が固有値に重解をもつ一組の例として挙げられる。\\
ちなみに、この時の固有値は、$\lambda_{1}=7=\lambda_{2}$、$\lambda_{3}=-7$となっている。

\vspace{1\baselineskip}

\end{document}











