\documentclass[a4j,10pt,oneside,openany]{jsbook}
%
\usepackage{amsmath,amssymb}
\usepackage{bm}
\usepackage{graphicx}
\usepackage{ascmac}
\usepackage{makeidx}
%
%\makeindex
%
\newcommand{\diff}{\mathrm{d}}  %微分記号
\newcommand{\divergence}{\mathrm{div}\,}  %ダイバージェンス
\newcommand{\grad}{\mathrm{grad}\,}  %グラディエント
\newcommand{\rot}{\mathrm{rot}\,}  %ローテーション
%
\setlength{\textwidth}{\fullwidth}
\setlength{\textheight}{44\baselineskip}
\addtolength{\textheight}{\topskip}
\setlength{\voffset}{-0.6in}
%
\title{{\Huge \textbf{Dr.Hongoの数理科学ゼミ 第165問}}\\}
\author{Yuji Hiramatsu}
\date{\today}
%
%
%
\begin{document}
%
%
\maketitle
%\frontmatter
%\tableofcontents
%
%
%\mainmatter

%\chapter{...}
%\begin{abstract}
%...
%...
%\end{abstract}

{\Huge 165問 解答}

\vspace{3\baselineskip}

{\Large (1)}
\\
\\
題意より、$f(n)$は、$_{n}C_{1},\;\; _{n}C_{2},\;\; _{n}C_{3},\;\; \dots ,\;\; _{n}C_{n},\;\; 0 $において、\\
初めて偶数になる項の番目であり、ここで、$f(n)=k$とおく。\\
また、同様に、$f(2n+1)$は、$_{2n+1}C_{1},\;\; _{2n+1}C_{2},\;\; _{2n+1}C_{3},\;\; \cdots ,\;\; _{2n+1}C_{2n+1},\;\; 0 $において、\\
初めて偶数となる項の番目であり、$f(2n+1)=2k$となることを証明すればよい。\\
ところで、$f(n)=k$の時、k番目の数列は、
\[ _{n}C_{k} = \frac{n!}{k! \; (n-k)!} = \frac{n \cdot (n-1) \cdot {n-2} \cdots (n-k+1)}{k \cdot (k-1) \cdots 2 \cdot 1} \]
と書くことができる。\\
ここで、分子及び分母の両方に、$2^{k}$をかけると、
\[ _{n}C_{k} = \frac{2n \cdot (2n-2) \cdot (2n-4) \cdots (2n-2k+2)}{2k \cdot (2k-2) \cdots 4 \cdot 2} \]
となる。\\
上式の分子に奇数、$(2n+1) \cdot (2n-1) \cdots (2n-2k+5) \cdot (2n-2k+3)$を、\\
分母に奇数、$(2k-1) \cdot (2k-3) \cdots 3 \cdot 1$をかけると、\\
\[ \frac{(2n+1) \cdot 2n \cdot (2n-1) \cdots (2n-2k+2)}{2k \cdot (2k-1) \cdots 3 \cdot 2 \cdot 1}=_{2n+1}C_{2k} \]
となり、$_{n}C_{k}$が偶数である以上は、$_{2n+1}C_{2k}$も偶数である。\\
$f(n)$に対応した数列においては、$k$番目より小さい項は奇数であり、$1\leqq i \leqq n-1$とし、上記と同様の操作をすると、\\
\[\frac{2^{i}n!}{2^{i}i!(n-i)!}\cdot\frac{(2n+1) \cdot (2n-1) \cdot (2n-3) \cdots (2n-2i+5) \cdot (2n-2i+3)}{(2i-1) \cdot (2i-3) \cdots 3 \cdot 1}= _{2n+1}C_{2i}\]
\[\frac{2^{i}n!}{2^{i}i!(n-i)!}\cdot\frac{(2n+1) \cdot (2n-1) \cdot (2n-3) \cdots (2n-2i+3) \cdot (2n-2i+1)}{(2i+1) \cdot (2i-1) \cdots 3 \cdot 1}= _{2n+1}C_{2i+1}\]
となる。
これらが、奇数であることは自明であり、また、$_{2n+1}C_{1}$は奇数であるので、
$f(2n+1)$に対応した数列において、初めて偶数となる項の番目$f(2n+1)$は、$2k$番目であることが証明された。

\vspace{1\baselineskip}

{\Large (2)}
\\
\\
(1)を繰り返し使用すると、$ \;\;  f(2015)= 2^{5} \cdot f(62) = 32 $

%\begin{thebibliography}{20}
% \bibitem{...}...
%\end{thebibliography}

\newpage
%\printindex
%
%
\end{document}