\documentclass[a4j,10pt,oneside,openany]{jsbook}
%
\usepackage{amsmath,amssymb}
\usepackage{bm}
\usepackage[dvipdfmx,hiresbb]{graphicx}
\usepackage{ascmac}
\usepackage{makeidx}
\usepackage{float}
\usepackage{wrapfig}
%
%\makeindex
%
\newcommand{\diff}{\mathrm{d}}  %微分記号
\newcommand{\divergence}{\mathrm{div}\,}  %ダイバージェンス
\newcommand{\grad}{\mathrm{grad}\,}  %グラディエント
\newcommand{\rot}{\mathrm{rot}\,}  %ローテーション
\newcommand{\btu}{\bigtriangleup} % triangle mark
\newcommand{\mr}{\mathrm}
%
\setlength{\textwidth}{\fullwidth}
\setlength{\textheight}{44\baselineskip}
\addtolength{\textheight}{\topskip}
\setlength{\voffset}{-0.6in}
%
\title{{\Huge \textbf{Dr.Hongoの数理科学ゼミ 第170問}}\\}
\author{Yuji Hiramatsu}
\date{}
%
%
%
\begin{document}
%
%
\maketitle
%\frontmatter
%\tableofcontents
%
%
%\mainmatter

%\chapter{...}
%\begin{abstract}
%...
%...
%\end{abstract}

{\Huge 170問 解答}

\vspace{3\baselineskip}

三角関数の有理関数の積分は、$t=\tan (\frac{\theta}{2})$と変数変換すれば、初等的に計算できることはよく知られた事実である。
$t=\tan (\frac{\theta}{2})$の時、
\begin{align*}
\sin \theta &= \frac{2t}{1+t^2}\\
\cos \theta &= \frac{1-t^2}{1+t^2}\\
d\theta &= \frac{2}{1+t^2}dt
\end{align*}
となるので、上式を$I$、$J$に代入し積分することを考える。\\
まず、$I$の積分を求める。上記の変数変換を利用すると、
\begin{align*}
I &= \int_{0}^{1} \; \frac{2 \cdot \frac{2t}{1+t^2}+2}{\frac{2t}{1+t^2}-2 \cdot \frac{1-t^2}{1+t^2}+5} \; \frac{2}{1+t^2} \; dt \\
  &= \int_{0}^{1} \; 4 \cdot \frac{t^2+2t+1}{(7t^2+2t+3)\cdot(t^2+1)} \; dt \\
  &= \frac{4}{5} \cdot \int_{0}^{1} \; \biggr{(} \frac{14t+2}{7t^2+2t+3}-\frac{2t}{t^2+1}+\frac{1}{t^2+1} \biggl{)} \; dt \\
  &= \frac{4}{5} \cdot \biggr{[} \log(7t^2+2t+3)-\log(t^2+1) \biggl{]} ^{1} _{0}+\frac{4}{5} \cdot \int^{\frac{\pi}{4}}_{0}  \; dx \\
  &= \underline{\frac{4}{5} \cdot (\log 2+\frac{\pi}{4})}
\end{align*}
となる。4行目第2項目の積分は、$t=\tan x$とすることでえられる。
$J$の積分も同様にして、
\begin{align*}
J &= \int_{0}^{1} \frac{2 \cdot \frac{1-t^2}{1+t^2}-4}{\frac{2t}{1+t^2}-2 \cdot \frac{1-t^2}{1+t^2}+5} \; \frac{2}{1+t^2} \; dt \\
   &= \int_{0}^{1} \; -4 \cdot \frac{3t^2+1}{(7t^2+2t+3)\cdot(t^2+1)} \; dt \\
   &= \frac{2}{5} \cdot \int_{0}^{1} \; \biggr{(} \frac{14t+2}{7t^2+2t+3}-\frac{2t}{t^2+1}-\frac{4}{t^2+1} \biggl{)} \; dt \\
   &= \frac{2}{5} \cdot \biggr{[} \log(7t^2+2t+3)-\log(t^2+1) \biggl{]} ^{1} _{0}-\frac{8}{5} \cdot \int^{\frac{\pi}{4}}_{0}  \; dx \\
   &= \underline{\frac{2}{5} \cdot (\log 2-\pi)}
\end{align*}
\\
ちなみに、題意より、$\frac{I}{2}+J=\int^{\frac{\pi}{2}}_{0} \; d\theta=\frac{\pi}{2}$となるはずであり、\\
上記結果を計算してみると、$\frac{I}{2}+J=\frac{\pi}{2}$になっていることが確認できる。

\end{document}











