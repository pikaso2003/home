\documentclass[a4j,10pt,oneside,openany]{jsbook}
%
\usepackage{amsmath,amssymb}
\usepackage{bm}
\usepackage{graphicx}
\usepackage{ascmac}
\usepackage{makeidx}
%
%\makeindex
%
\newcommand{\diff}{\mathrm{d}}  %微分記号
\newcommand{\divergence}{\mathrm{div}\,}  %ダイバージェンス
\newcommand{\grad}{\mathrm{grad}\,}  %グラディエント
\newcommand{\rot}{\mathrm{rot}\,}  %ローテーション
\newcommand{\btu}{\bigtriangleup} % triangle mark
\newcommand{\mr}{\mathrm}
%
\setlength{\textwidth}{\fullwidth}
\setlength{\textheight}{44\baselineskip}
\addtolength{\textheight}{\topskip}
\setlength{\voffset}{-0.6in}
%
\title{{\Huge \textbf{Dr.Hongoの数理科学ゼミ 第166問}}\\}
\author{Yuji Hiramatsu}
\date{}
%
%
%
\begin{document}
%
%
\maketitle
%\frontmatter
%\tableofcontents
%
%
%\mainmatter

%\chapter{...}
%\begin{abstract}
%...
%...
%\end{abstract}

{\Huge 166問 解答}

\vspace{3\baselineskip}

{\Large (1)}
\\
\\
RCとVQは並行であるので、$\btu$RCB $\sim$ $\btu$ VQB である。\\
よって、$\mr{RC}:\mr{VQ} = 3:1$であり、
一方で、$\mr{AC}:\mr{RC} = 3:1$なので、\\
\[ \underline{\mr{AC}:\mr{VQ} = 9:1} \]
また、$\btu$ART $\sim$ $\btu$ QVT であるので、\\
\[ \underline{\mr{AT}:\mr{TQ} = \mr{AR}:\mr{VQ} = 6:1} \]

\vspace{1\baselineskip}

{\Large (2)}
\\
\\
$\btu$PCB $\sim$ $\btu$ WQB であるので、$\mr{PW} : \mr{WB} = 2:1$である。\\
一方で、$\mr{AP}:\mr{PB} = 1:2$なので、$\mr{AP}:\mr{PW} : \mr{WB} = 3:4:2$である。\\
よって、
\[ \underline{\mr{AP} : \mr{PW} = 3:4} \]
$\btu$ASP $\sim$ $\btu$ AQW なので、$\mr{AS}:\mr{SQ} = 3:4$ \\
また、(1)より、$ \mr{AT}:\mr{TQ} = 6:1 $であったので、
$ \mr{AS} : \mr{ST} : \mr{TQ} = 3 : 3 : 1 $
である。\\
よって、
\[ \underline{\mr{AS}:\mr{ST} = 1:1} \]

\vspace{1\baselineskip}

{\Large (3)}
\\
\\
$\btu$ABCの面積を $S$ とすれば、\\
\[ {\btu}\mr{AQC} = \frac{2}{3} {\btu}\mr{ABC} = \frac{2}{3}S \]
であり、(2)より、
\[ {\btu}\mr{SCT} = \frac{2S}{3} \; \frac{3}{7} = \frac{2}{7}S \]
と分かる。
点R、Q、Pはそれぞれが存在する辺を$1:2$に内分しており、対称性から、\\
$\mr{CU}:\mr{US} = 1:1$である。\\
以上から、
\[ {\btu}\mr{STU} = \frac{2S}{7} \; \frac{1}{2} = \frac{1}{7}S = 1 \]
である。\\
よって、求める答えは、$\underline{1}$
%\begin{thebibliography}{20}
% \bibitem{...}...
%\end{thebibliography}

\newpage
%\printindex
%
%
\end{document}











